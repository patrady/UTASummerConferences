The University of Texas at Arlington hosts summer camps during the summer months of May, June, July, and August ranging from organizational to religious camps. To register, accommodate their stay, and bill these camps, all their data is stored on Google Sheets in a variety of tabs. This include contract information, rosters, checked out equipment, stay durations, parking, work orders, and billing. By manipulating permissions to establish a five-tier hierarchy, that of administrators, professional staff, residence directors, resident assistants, and camp coordinators, where each group will have different permissions within the set of sheets. Since the data stored in Google Sheets is not covered under FERPA, it can be legally stored in this public manner but is encountering several issues. First, there can be upwards to thirty employees, masked behind their respective titles, touching the Google Sheet without individually identifying themselves. Therefore, data can be edited and deleted at whim without knowing who was responsible for it. Second, human error has led to an estimated loss of 50,000 over the past three months. This includes incorrect check-in/check-out dates for campers, forgetfulness in duplicate manual entries, and an inability to utilize all open residence hall rooms. Third, staff members are triple checking both the inputted data in Google Sheets and their hall cards, keys, and temporary cards to ensure that they are in order whereas most of these processes can be automated. Lastly, the administration spends several hours a week managing permissions for each sheet (it is not unusual to have up to 15 sheets per camp), ensuring that the five-tier hierarchy remains in place for each camp.
The business case for this system will begin when a camp requests to stay at UTA. The gathering of basic information, such as the number of bed spaces, duration, extra meeting spaces, dining, linens, etc., will be collected. Upon preliminary approval, legal paperwork will need to be filled out that outlines a contract between the camp director and UTA. Just prior to the camp arriving, the roster including all names of the campers will need to be uploaded. On check-in, campers, camp counselors, and camp coordinators will need to be assigned to a specific bed space, given a respective magnetic strip card and hard key for their room and accommodate parking. It is possible that during a stay, a maintenance emergency happens such that multiple campers may need to move rooms and a work order needs to be submitted. Additionally, campers may want to check out recreational equipment for billiards, table tennis, or video game consoles. If any of these items are damaged or lost, these details should be noted. Upon check-out, all magnetic cards, hard keys, and card holders need to be returned and assessed for damages. Any keys lost or damaged should be noted for billing purposes. After a camp has completed their stay, the used rooms need to be checked for any damages, cleaned, and finally set back up for the following camp. Any damages found inside the rooms will need to be noted for billing purposes. Finally, once it is time to bill a camp, a grand total will be calculated from the number of used bed spaces, their duration, parking, damaged equipment, damaged cards, extraneous meeting spaces, dining, linens, etc. If desirable, a payment confirmation can be uploaded for filing purposes.
This project will add tremendous value to UTA Summer Conferences, allowing for less human error, increased efficiency, eliminating several manual processes, and increasing UTA’s professional public appearance. As this project will come at no charge to their organization, Summer Conferences is excited to have a custom system that will accommodate all of their needs.
